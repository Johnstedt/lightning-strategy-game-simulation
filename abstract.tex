\begin{abstract}
    In camera pre-calibration, images of a calibration object are commonly 
    used to determine the internal geometry of a camera.
    The calibration imaging is often optimized to have the calibration object 
    cover as large image area as possible.
    This is likely to yield a larger concentration
    of measured image points near the center of the image sensor.
    In this report, the hypothesis is investigated that this 
    non-uniform image point distribution results in a sub-optimal calibration.
    An area-based re-weighting scheme is suggested to improve the calibration.
    Additionally, the effect of a choice between a 2D and a 3D calibration
    object is investigated.

    A simulation study was performed where both a standard and area-weighted
    pre-calibration scheme was used in a parallel and a convergent scene.
    The estimated uncertainty and true errors were computed
    and compared to the first order predictions and results of perfect calibrations.
    The area-based calibration showed no reduction in estimation errors.
    Furthermore, the 3D calibration object did not give a noticeable improvement.
    However, for the standard and area-based calibrations, the true errors
    surpassed the estimated uncertainties by up to
    26 and 58 percent, respectively.
\end{abstract}
