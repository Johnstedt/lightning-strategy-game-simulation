\section{Lightning Network}
\label{sec:lightning:network}

The Lightning Network is a network of micro-payment channels on top of the Bitcoin Network. The network was first conceived and defined in 2014\cite{poon:dryja:lightning:network}, became viable with the segregated witness upgrade\cite{bip:0141:segwit} and got activated through a user-activated-soft-fork in mid 2017\cite{bip:uasf:segwit}. The upgrade moves(segregates) the unlocking script(witness) from the transaction which fixes transaction malleability since multiple unlocking scripts can be valid for a single locking script. 

The network protocol have been defined\cite{repository:lightning:rfc} with four independent actors implementing protocol compliant nodes\cite{repository:lnd}\cite{repository:eclair}\cite{repository:clightning}\cite{repository:lit}.

\subsection{Payment channels}

The Lightning network consists of many payment channels between both routing and non-routing nodes.
A payment channels is a essentially a transaction containing funds locked up on the bitcoin network. It requires signatures from both parties to be able to spend the transaction and cease the channel.

Initially each party have a signed transaction splitting the funds to their original committed amount\footnote{In the RFC each channel is funded by only one party so the split is always all to one party.}. The parties can then agree on a different balance signing a new transaction invalidating the previous spending transaction essentially allowing for infinite transactions inside the channel without any burden on the bitcoin network.

\subsubsection{Funding Transaction}

A payment channel is funded by a funding transaction defined in the lightning paper as such\cite{poon:dryja:lightning:network}:
\begin{enumerate}
	\item  Create the parent (Funding Transaction)
	\item  Create the children (Commitment Transactions and all spends from the commitment transactions)
	\item  Sign the children
	\item  Exchange the signatures for the children
	\item  Sign the parent
	\item  Exchange the signatures for the parent
	\item  Broadcast the parent on the blockchain
\end{enumerate}

It is critical that the signatures of the spending transaction is exchanged before the funding transaction otherwise the funds could be held hostage by an uncooperative partner\footnote{This is possible with the \texttt{SIGHASH\_NOINPUT} transaction defined in BIP118\cite{bip:0118:sighash:noinput} which decouples the signature from the specific transaction hash of the output.}. By exchanging the spending transaction first either party could broadcast it to receive the initial funds.

\subsection{Multihop payments}

