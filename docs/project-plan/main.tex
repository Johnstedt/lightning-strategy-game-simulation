\documentclass[10pt, titlepage, oneside, a4paper]{article}
\usepackage[english]{babel}
\usepackage[T1]{fontenc}
\usepackage[titletoc]{appendix}
\usepackage[utf8]{inputenc}
\usepackage{amssymb, graphicx, fancyhdr}
\usepackage{amsmath}
\usepackage{amsthm,algorithm,yhmath,enumitem,lscape}
\usepackage[noend]{algpseudocode}
\usepackage{pifont}
\usepackage{tcolorbox}
\usepackage{listings}
\usepackage{float}
\usepackage{subcaption} 
\usepackage{amstext}
\usepackage{graphicx}
\usepackage{pdfpages}
\usepackage{pgfplots}
\usepackage{hyperref}

\usepackage{caption}




%\usepackage{color} %red, green, blue, yellow, cyan, magenta, black, white
\definecolor{mygreen}{RGB}{28,172,0} % color values Red, Green, Blue
\definecolor{mylilas}{RGB}{170,55,241}

%\usepackage[table,xcdraw]{xcolor}
%\usepackage[normalem]{ulem}
%\useunder{\uline}{\ul}{}

\usepgfplotslibrary{external}
\pgfplotsset{compat=1.10}
\tikzexternalize 
\makeatletter
\def\verbatim{\tiny\@verbatim \frenchspacing\@vobeyspaces \@xverbatim}
\makeatother

% Define a \HEADER{Title} ... \ENDHEADER block
\makeatletter
\newcommand{\HEADER}[1]{\ALC@it\underline{\textsc{#1}}\begin{ALC@g}}
\newcommand{\ENDHEADER}{\end{ALC@g}}
\makeatother
\usepackage{float}
\usepackage{courier}
\usepackage{hyperref, url}
\usepackage{hyperref,xcolor}% http://ctan.org/pkg/{hyperref,xcolor}
\definecolor{wine-stain}{rgb}{0.5,0,0}
\hypersetup{
  colorlinks,
  linkcolor=black,
  linktoc=all
}

\addtolength{\textheight}{20mm}
\addtolength{\voffset}{-5mm}
\renewcommand{\sectionmark}[1]{\markleft{#1}}

\addtolength{\textheight}{20mm}
\addtolength{\voffset}{-5mm}
\renewcommand{\sectionmark}[1]{\markleft{#1}}

\def\inst{}
\def\course{ Routing Fee Estimation in the Lightning Network }
\def\title{Project Plan}
\def\name{John-John Markstedt}
\def\username{}
\def\email{\username{c14nbn}@cs.umu.se}
\def\path{FIXA THE GOD DAMN SÖKPATH}

\def\mfullpath{\raisebox{1pt}{$\scriptstyle \sim$}\musername/\path}
\def\nfullpath{\raisebox{1pt}{$\scriptstyle \sim$}\nusername/\path}

\newcommand{\R}{\mathbb{R}}
\newcommand{\N}{\mathbb{N}}
\newcommand{\Rn}{\mathbb{R}^n}
\newcommand{\Rnn}{\mathbb{R}^{n \times n}}
\newcommand{\bes}{\begin{equation*}}
\newcommand{\ees}{\end{equation*}}
\newcommand{\be}{\begin{equation}}
\newcommand{\ee}{\end{equation}}
\newcommand{\bms}{\begin{multline*}}
\newcommand{\emults}{\end{multline*}}
\newcommand{\bbm}{\begin{bmatrix}}
\newcommand{\ebm}{\end{bmatrix}}
\newcommand{\eps}{\epsilon}
\newcommand{\fl}{\text{fl}}
\newcommand{\Lp}{{L^p}}
\newcommand{\Ker}{\text{Ker}\,}
\newcommand{\loc}{{\text{loc}}}
\newcommand{\ccinf}{C_c^\infty}
\newcommand{\supp}{\text{supp}}
\newcommand{\dist}{\text{dist}}


\begin{document}
\setcounter{secnumdepth}{5}
	\begin{titlepage}
		\thispagestyle{empty}
		\begin{large}
			\begin{tabular}{@{}p{\textwidth}@{}}
				\textbf{UMEÅ UNIVERSITY \hfill \today} \\
				\textbf{Department of Computing Science} \\
			
			\end{tabular}
		\end{large}
		\vspace{25mm}
		\begin{center}
			
			\huge{\textbf{\course}}\\
			\vspace{10mm}
			\LARGE{\title} \\
			\vspace{2mm}
		    \LARGE{preliminary 1.0} \\
            \vspace{10mm}
			\begin{large}
				\begin{tabular}{ll}
                	\textbf{Name} & \name \\
					\textbf{Cinnober Supervisor} & Oskar Janson\\
					\textbf{University Supervisor} & Jerry Eriksson\\
					\textbf{Examinator} & Henrik Björklund\\
				\end{tabular}
			\vfill
            \vfill			\end{large}


		\end{center}
	\end{titlepage}
    
    % fixar sidfot
	\rfoot{\footnotesize{\today}}
	\lhead{\sc\footnotesize\title}
	\rhead{\nouppercase{\sc\footnotesize\leftmark}}
	\pagestyle{fancy}
	\renewcommand{\headrulewidth}{0.2pt}
	\renewcommand{\footrulewidth}{0.2pt}

	% skapar innehållsförteckning.
	% Tänk på att köra latex 2ggr för att uppdatera allt
	% och lägger in en sidbrytning
    \pagenumbering{roman}
	\newpage

	\pagenumbering{arabic}

	% i Sverige har vi normalt inget indrag vid nytt stycke
	\setlength{\parindent}{0pt}
	% men däremot lite mellanrum
	\setlength{\parskip}{5pt}
\tableofcontents
\newpage

\section{Introduction}

This is the project plan for the master thesis of Routing Fee Estimation in the Lightning Network. Some parts are directly taken from the project specification to make it a self contained document.

\subsection{Background}

Although still very much in its early stages, blockchain and distributed ledger
solutions are said to be able to transform the current financial infrastructure,
especially the post-trade side. Most notable of all applications of blockchain
technology is Bitcoin, which is an open decentralized network with an effectively
immutable database of transactions, shared by all full nodes in the network. The
internal currency, bitcoins, is provided to miners who help secure the network by
participating in the Proof of Work consensus algorithm. Since Bitcoin’s nascence,
other similar solutions have appeared. Some are modified forks of Bitcoin while
some are built from scratch using the same ideas.
One of the challenges that all public decentralized blockchain networks faces is
how it should handle an increased number of transactions. Increasing throughput
is important in order to make the network capable of supporting e.g. regular
payments. There are various ways to increase throughput, but roughly they sort
into two categories:
\begin{itemize}
	\item Increase throughput on the base layer, with consequences of increased requirements for participation in the base layer as a full node.
	\item  Create layers on top of the base layer where most updates happen, only
	resolving the sum of updates to the layer below.
\end{itemize}

In 2017, the Bitcoin scaling debate raged on, in the end resulting in a split of
both community and coin. In the aftermath a fork of Bitcoin was created, called
Bitcoin Cash. Bitcoin Cash had the approach that throughput should increase in
the base layer, in the way of increasing block size. Bitcoin itself had a more
restrained strategy for the base layer, favoring the additional layers approach.
Specifically, payment channels and the Lightning Network would provide an
outlet for smaller transactions in the network.
In a payment channel, a set amount of bitcoin is committed for use by the
senders and the channel is created by an opening transaction. Once the
transaction has been registered, parties in the channel that have a positive

\subsection{Problem Description}

In order to get compensation for providing channels for routing, the routing node
can decide to take a fee for the use of its channels. One of the problems a node
operator faces is what the fair price of routing through its channels is. A too high
fee would result in the channel being ignored and payments would be routed
through other nodes with competitive fees. It does not want to be too cheap
either, since running a routing node carries risk (nodes are always online and are
a target for attacks) and has requirements in terms of hardware and time spent.
The task then comes to finding a competitive fee that is fair but profitable.
Examples on factors that can be considered:
\begin{itemize}
	\item Competing fees for routing towards the same segment of the network.
	\item The cost for the operator in rebalancing their channels.
	\item Operating costs for running the node.
\end{itemize}
The main problem to solve in this thesis project is that of finding the optimal fee
for a routing node.
Further, the optimal strategy for operating a routing node should be
investigated. Sample questions to answer are:
\begin{itemize}
	\item What is the fair discount for a transaction that balances the channel, as
	opposed to an unbalancing one?
	\item When should a new channel be opened, and what node should the new
	channel connect to in order to incentivize routing through the node?
	\item When should a manual rebalancing of the channels occur, as opposed to
	incentivized rebalancing using discounts?
\end{itemize}

\subsection{Goal}

The goal of the thesis is to investigate and implement models for the
estimation of routing fees for a routing node on the lightning network.

\section{Objectives}

These are the main objectives with the project:

\begin{itemize}
	\item A study of payment channels, the current Lightning Network and
	proposed future improvements.
	\item A study of routing node operation, what are the actions, costs and risks
	that need to be considered when operating a node?
	\item A study of techniques useful for estimating the routing fee.
	\item A model for estimating the routing fee should be produced.
\end{itemize}

\section{Literature and references}

The literature and references listed below will make up the base of the thesis. More literature and references will be added as the project progresses. 

Andreas Antonopoulos book Mastering Bitcoin\cite{antonopoulos:mastering:bitcoin} is the book on bitcoin and will provide a sound basis for the on-chain necessities.

The original bitcoin\cite{nakamoto:bitcoin} and lightning\cite{Poon:Dryja:lightning:network} papers and all proposed protocol improvements\cite{bip} are great primary sources.

The bitcoin reference implementation\cite{repository:bitcoin}, along with other node implementations\cite{repository:neutrino}\cite{repository:btcd} as well as lightning specification\cite{repository:lightning:rfc} and implementations\cite{repository:lnd}\cite{repository:clightning}\cite{repository:eclair} will be studied and used during the project.



\section{Communication}

In this section the various aspects of communication during the project is discussed.

\subsection{Weekly report}

A report in the form of a markdown document will be written each week and be available in the thesis repository\cite{repository:me:thesis}. Each document will roughly follow the same structure with three major categories, namely, What I've been doing during the week, Problems I'm faced with going forward and what I should be doing the following week.

\subsection{Documents and thesis report}

All the project's resources will be available through a git repository\cite{repository:me:thesis}. There the project's progression can be observed. Git is a version control system, this will provide an extensive history over the project. Every type of resource will be made available through git including text, images and source code. 

\subsection{Meetings}

As the thesis will be written from the Cinnober office, meetings with the Cinnober supervisor Oskar Janson can easily be had when needed.

<Some plan with Jerry>


\section{Time plan}

An external timeplan for the project should most likely be attached alongside this document. Otherwise it shall always be available in the thesis repository\cite{repository:me:thesis}. The timeline contains deadlines set be the university as well as a rough estimation on the project tasks.




\bibliographystyle{plain}
\bibliography{references.bib}

\end{document}
