\chapter{The Lightning Network}
\label{sec:lightning:network}

The \gls{Lightning Network} is a network of micro-payment \gls{channel}s on top of the Bitcoin Network. The network was first conceived and defined in 2014~\cite{poon:dryja:lightning:network}, became viable with the segregated witness upgrade~\cite{bip:0141:segwit} and was activated through a \textit{user-activated-soft-fork} in mid 2017~\cite{bip:148:uasf:segwit}. The upgrade moves(segregates) the unlocking script(witness) from the transaction which fixes transaction malleability since multiple unlocking scripts may be valid for a single locking script. 

The network protocol has been defined~\cite{repository:lightning:rfc} with four independent actors implementing protocol compliant nodes~\cite{repository:lnd, repository:eclair, repository:clightning, repository:lit}.

\section{Payment channels}

The Lightning Network consists of many payment channels between both routing and non-routing \gls{node}s.
A payment channel is essentially a transaction containing funds locked up on the bitcoin network. It requires signatures from both parties to be able to spend the transaction and cease the channel.

Initially each party has a signed transaction splitting the funds to their original committed amount\footnote{In the RFC each channel is funded by only one party so the initial split is always all to one party.}. The parties can then agree on a different balance signing a new transaction invalidating the previous spending transaction essentially allowing for infinite transactions inside the channel without any burden on the bitcoin network.

\subsection{Funding Transaction}

A payment channel is funded by a funding transaction defined in the lightning paper by following steps~\cite{poon:dryja:lightning:network},

\begin{enumerate}
	\item  Create the parent (Funding Transaction)
	\item  Create the children (Commitment Transactions and all spends from the commitment transactions)
	\item  Sign the children
	\item  Exchange the signatures for the children
	\item  Sign the parent
	\item  Exchange the signatures for the parent
	\item  Broadcast the parent on the blockchain
\end{enumerate}

It is critical that the signatures of the spending transaction are exchanged before the funding transaction otherwise the funds could be held hostage by an uncooperative partner\footnote{This is possible with the \texttt{SIGHASH\_NOINPUT} transaction defined in BIP118~\cite{bip:0118:sighash:noinput} which decouples the signature from the specific transaction \gls{hash} of the output. This is not yet activated and only single source funding is currently available}. By exchanging the spending transaction first either party could broadcast it to retrieve the initial funds.

\subsection{Commitment Transaction}

In order for a channel to be useful the creation of a new \gls{commitment transaction}\footnote{It may be easier to think of these types of transaction outputs as smart contracts, as in this context it's the same thing.}, updating the balance, would need to invalidate the previous one. The previous transaction is after all still a valid bitcoin transaction. This creates a problem since each party would benefit from different commitment transactions being broadcast to the network.

The solution is to construct the \gls{commitment transaction} similar to that of a fidelity bond where one party would be penalized if violating the agreement. In this case violation is broadcasting any but the latest commitment transaction. To be able to ascribe blame to the violating party the commitment transactions for the same channel state must be unique as otherwise one would not know which party broadcast the transaction to the network. Only uniqueness is however not enough since there is no way to enforce the penalty for violation.

The enforcement is possible if revocation is built into the commitment transaction, where the other party could revoke a faulty broadcast. This became possible with the addition of relative block maturity introduced with BIP68(see section \ref{sec:rlt}). The commitment transaction has two outputs, one is the other party's funds spendable immediately. The other output has two redeemable paths. Either,

\begin{enumerate}
	\item The broadcasting party can spend the funds after a defined relative block time has passed. Essentially leaving time window for dispute for the violated party to enforce the penalty(Revocable delivery).
	\item The non broadcasting party can refute that the commitment transaction is not last signed and spend the output as penalty.
\end{enumerate}

Remember that the commitment transactions is reversed for the other party. This type of output which enable revocation is called a Revocable Sequence Maturity Contract(\gls{RSMC}).

\begin{figure}[!htb]
			\hspace*{-0.8cm} 
	\centering
	\includegraphics[width=8.5cm]{images/RSMC.png}
	\caption{\textit{Commitment transactions. 
	}}
	\label{fig:commitment:tx}
			\hspace*{2mm} 
\end{figure}

\subsection{Revocation of old commitment transaction}

In order to update the channel balance a new pair of commitment transactions is created. The previous commitment transactions is invalidated by both signing a Breach Remedy Transaction(\gls{BRT}) spending from their own previous \gls{commitment transaction}s \gls{RSMC} output and handing it over to the other party. If a previous commitment transaction were to be broadcast the BRT would take precedence since the revocable delivery takes a certain amount of time before it is valid. The Breach Remedy Transaction will send all funds to the non faulty party, this is the penalty for violating the contract.

It is necessary for each party to monitor the \gls{blockchain} for old commitment transactions and to broadcast the \gls{BRT} in case it happens. If the BRT is not broadcast in the designated time the revocable delivery output of the previous commitment transaction will become valid and may be spent, settling the incorrect channel balance is settled and valid on the blockchain.

\subsection{Alice and Bob opens a channel}

Alice and Bob wants to open a channel. They agree on funding the channel with \textbf{0.5\bitcoin} each. They construct a funding transaction $F$ which require both Alice and Bobs signatures to be spent. Alice constructs a $CT_{A1}$ with two outputs $RSMC_{A1}$ and $SO_{B1}$ and similarly Bob constructs $CT_{B1}$ with outputs $RSMC_{B1}$ and $SO_{A1}$. Alice signs $CT_{B1}$ and Bob signs $CT_{A1}$\footnote{A note on notation - The lowered A,B indicates whether the output is spendable by Alice or Bob. The number denote the channel state the output corresponds to. SO stands for Spendable Output.}. Alice and Bob exchange signatures for $F$ and either of them broadcast $F$.

Alice wants to buy a cookie from Bob for \textbf{0.2\bitcoin} and Bob agrees to sell. They each constructs new \gls{commitment transaction}s $CT_{A2}$, $CT_{B2}$, each with new $RSMC_{A2}$, $RSMC_{B2}$ and $SO_{B2}$, $SO_{A2}$ that reflects the new balance \textbf{0.7\bitcoin} to Bob and \textbf{0.3\bitcoin} to Alice without signing them. Before Bob can hand Alice the cookie the old $CT_{A1}$, $CT_{B1}$ is invalidated by Alice signing a $BRT_{B1}$ spending from $RSMC_{A1}$. Although Bob has nothing to gain by broadcasting $CT_{B1}$ as he would net \textbf{0.2\bitcoin} less - Bob signs $BRT_{A1}$ spending from $RSMC_{B1}$. Alice signs $CT_{B2}$ and Bob signs $CT_{A2}$. The channel balance is updated.

If Alice tries to undo the cookie purchase by broadcasting $CT_{A1}$ giving \textbf{0.5\bitcoin} each. Bob simply broadcasts the $BRT_{B1}$ spending from Alice's share of the $CT_{A1}$ and Bob receives \textbf{1\bitcoin} and Alice \textbf{0\bitcoin}. 

As long as the channel is open Alice and Bob may settle any imbalances between them with a new commitment transaction. 

\begin{figure}[!htb]
		\hspace*{-1.2cm} 
	\centering
	\includegraphics[width=9cm]{images/RSMC_flow.png}
	\caption{\textit{Showing the channel balance state between Alice and Bob after the second Commitment transaction has been made. 
	}}
	\label{fig:rsmc_second_commitment}
		\hspace*{2mm} 
\end{figure}

\subsection{Contracts and implementation}
\label{sec:hdw}
Signing and sending $\gls{BRT}$s is unnecessarily inefficient if a HD wallet is used to generate keys\cite{bip:0032:hd:wallet}. HD wallets have a master key which can derive new keys in a tree structure. Each key in the tree can derive its children and no child can derive its parent. This works both for the public and the private keys independent of each other. A new key is used for each $CT$ and if that is deep in the tree, say a millionth layer down, and can simply be disclosed to the other party whenever the $CT$ is invalidated. The public master key may be shared with the other party so all public keys can be derived for transaction construction. With each new invalidated $CT$ the previous $CT$'s parent key may be used. So each party only needs to store the latest disclosed key as the other child keys can be derived from it.

\begin{figure}[hbt!]
	\centering
	\begin{lstlisting}[
		basicstyle=\small, %or \small or \footnotesize etc.
	]
<@\texttt{\textcolor{black}{OP\_IF}}@>
  <@\texttt{\textcolor{black}{<time> OP\_CHECKSEQUENCEVERIFY OP\_DROP}}@>
  <@\texttt{\textcolor{black}{OP\_HASH160 <A pk> OP\_EQUALVERIFY OP\_CHECKSIG}}@>
<@\texttt{\textcolor{black}{OP\_ELSE}}@>
  <@\texttt{\textcolor{black}{2 <A r pk><B r pk> 2 OP\_CHECKMULTISIGVERIFY}}@>
<@\texttt{\textcolor{black}{OP\_ENDIF}}@>
	\end{lstlisting}
	
	\caption{\textit{ A Revocable Sequence Maturity Contract.
	}}
	\label{fig:RSMC}
\end{figure}

Consider a Revocable Sequence Maturity Contract seen in Figure \ref{fig:RSMC}. It is constructed by Alice and she used Bob's master public key to derive Bob's revocation public key in some prearranged manner. To invalidate the $CT$, Alice need only to disclose the private key corresponding to her revocation public key for Bob to be able to revoke a violation. 

\subsection{Derived revocation address}

The 2 of 2 multi-signature clause paying out to the violated party may be implemented in a denser form by paying to a blinded key neither party control. The private key is then derived from knowledge split between the two parties. When a new $CT$ is created the secret is shared so only the violated party could spend from the revocation clause in case the old $CT$ is broadcast. So hence forth only a revocation pubkey is used to denote a revocation output.

\subsection{Closing channels}

It's possible to close a channel by broadcasting either of the latest $CT$s. However the broadcasting needs to wait the designated time period, losing out on the time value of money. Instead both parties may agree to construct a new Exercise Settlement Transaction($EST$) with direct spendable outputs instead of a $RSMC$ output. Signing the $EST$ would close the channel and allowing both parities immediate access to the funds.

\section{Network wide payments}

By combining channels into a network, a payment could potentially ripple through multiple channels and may allow parties without a direct channel to transact. Such a multi-hop payment would need the same trust-less custodial properties for the routing middle \gls{node}s as between direct channel parties shown above. This is solved by utilizing a Hashed Timelock Contract(\gls{HTLC}) for each of the channels in the route.

\subsection{Hashed Timelock Contract}

A $\gls{HTLC}$ is an extra output added to a $CT$ which enforces a payment from Alice to party Bob if and only if Bob can produce and disclose a preimage R of a \gls{hash} in a designated time period. 

\begin{figure}[hbt!]
	\centering
	\begin{lstlisting}[
	basicstyle=\small, %or \small or \footnotesize etc.
	]
<@\texttt{\textcolor{black}{OP\_IF}}@>
  <@\texttt{\textcolor{black}{OP\_HASH160 <$H_R$> OP\_EQUALVERIFY}}@>
  <@\texttt{\textcolor{black}{2 <A $PK_2$><B $PK_2$> OP\_CHECKMULTISIGVERIFY}}@>  
<@\texttt{\textcolor{black}{OP\_ELSE}}@>
  <@\texttt{\textcolor{black}{<time> OP\_CHECKSEQUENCEVERIFY OP\_DROP}}@>
  <@\texttt{\textcolor{black}{2 <A $PK_1$><B $PK_1$> 2 OP\_CHECKMULTISIGVERIFY}}@>
<@\texttt{\textcolor{black}{OP\_ENDIF}}@>
	\end{lstlisting}
	
	\caption{\textit{ A Hashed Timelock Contract. It's not yet viable for payments in this state.
	}}
	\label{fig:HTLC}
\end{figure}

A $HTLC$ can be seen in Figure \ref{fig:HTLC} and contains two validation paths. 

\begin{enumerate}
	\item If Bob has the knowledge of R that produces $H_{R}$ he can validate the first path. (Execution)
	\item If Bob fails to prove knowledge of preimage R, Alice may validate the second path after the designated time period has passed. (Timeout)
\end{enumerate}

\subsection{Revocation of HTLCs}

$\gls{HTLC}$s have a similar problem as with direct payment channels, if an old $CT$ is broadcast both validation paths may be valid. In the same manner as before, the $HTLC$ is made revocable by the construction of two different contracts for each party $HTLC_{A1}$ and $HTLC_{B1}$. So here again the possibility to ascribe blame exists. Adding revocable paths to $HTLC_{A1}$ and $HTCL_{B1}$ is necessary\footnote{It's necessary to include this path as an uncooperative party may never broadcast the execution/timeout transaction in which revocation is possible.} but not enough since the execution path is also valid. A period for dispute must still be had as the disclosure may be correct but may be from an old commitment transaction. The same is true for the timeout path as Bob may not have knowledge of $R$ in an old commitment transaction.   

Instead the timeout / execution path to oneself is spent into new transactions with a dispute period and ability for the other party to revoke an incorrect broadcast. 

\begin{figure}[hbt!]

	\centering

	\begin{lstlisting}[
	basicstyle=\small, %or \small or \footnotesize etc.
	linewidth=10cm
	]
<@\texttt{\textcolor{black}{OP\_DUP OP\_HASH160 <$H_{RPK}$> OP\_EQUAL}}@>
<@\texttt{\textcolor{black}{OP\_IF}}@>
  <@\texttt{\textcolor{black}{OP\_CHECKSIG}}@>
<@\texttt{\textcolor{black}{OP\_ELSE}}@>
  <@\texttt{\textcolor{black}{<$PK_{B}$> OP\_SWAP OP\_SIZE 32 OP\_EQUAL}}@>
  <@\texttt{\textcolor{black}{OP\_NOTIF}}@>
    <@\texttt{\textcolor{black}{OP\_DROP 2 OP\_SWAP <$PK_{A}$> 2 OP\_CHECKMULTISIG}}@>  
  <@\texttt{\textcolor{black}{OP\_ELSE}}@> 
    <@\texttt{\textcolor{black}{OP\_HASH160 <$R_{preimage}$> OP\_EQUALVERIFY}}@> 
  <@\texttt{\textcolor{black}{OP\_ENDIF}}@>
<@\texttt{\textcolor{black}{OP\_ENDIF}}@>
	\end{lstlisting}
	
	\caption{\textit{ The $HTLC_{A1}$ as offered by Alice
	}}
	\label{fig:alice:HTLC}
\end{figure}

The $HTLC_{A1}$ seen in Figure \ref{fig:alice:HTLC} can only be broadcast by Alice and contains three validation paths:

\begin{enumerate}
	\item Bob can revoke if it's not the last $CT$ constructing a $\gls{BRT}$ spending the output.
	\item The output is spent by a pre-signed transaction to a new transaction $ST$ with a dispute period.
	\item Bob can disclose knowledge of R and spend the output.
\end{enumerate}

The order of the two last clauses is reversed for Bob's $HTLC_{B1}$ as if Bob can prove knowledge of $R$ the output is spent to a new transaction $ST$ for disputation.

\begin{figure}[hbt!]
	
	\centering
	
	\begin{lstlisting}[
	basicstyle=\small, %or \small or \footnotesize etc.
	linewidth=10cm
	]
<@\texttt{\textcolor{black}{OP\_DUP OP\_HASH160 <$H_{RPK}$> OP\_EQUAL}}@>
<@\texttt{\textcolor{black}{OP\_IF}}@>
  <@\texttt{\textcolor{black}{OP\_CHECKSIG}}@>
<@\texttt{\textcolor{black}{OP\_ELSE}}@>
  <@\texttt{\textcolor{black}{<$PK_{B}$> OP\_SWAP OP\_SIZE 32 OP\_EQUAL}}@>
  <@\texttt{\textcolor{black}{OP\_IF}}@>
    <@\texttt{\textcolor{black}{OP\_HASH160 <$R_{preimage}$> OP\_EQUALVERIFY}}@> 
    <@\texttt{\textcolor{black}{OP\_DROP 2 OP\_SWAP <$PK_{A}$> 2 OP\_CHECKMULTISIG}}@>  
  <@\texttt{\textcolor{black}{OP\_ELSE}}@> 
    <@\texttt{\textcolor{black}{OP\_DROP <time> OP\_CHECKLOCKTIMEVERIFY OP\_DROP}}@> 	
  <@\texttt{\textcolor{black}{OP\_ENDIF}}@>
<@\texttt{\textcolor{black}{OP\_ENDIF}}@>
	\end{lstlisting}
	
	\caption{\textit{ The $HTLC_{B1}$ as offered by Bob
	}}
	\label{fig:bob:HTLC}
\end{figure}

The $HTLC_{B1}$ seen in Figure \ref{fig:bob:HTLC} thus contain:

\begin{enumerate}
	\item Alice can revoke if it's not the last $CT$ constructing a $BRT$ spending the output.
	\item Bob can prove knowledge of $R$ the output is spent to a new transaction $ST$ with a dispute period.
	\item After the designated time Alice can spend the output.
\end{enumerate}

The new transactions by which both the second clauses spends to are constructed and pre-signed before the $HTLC$s is exchanged simply adds time for revocation as seen for $ST_{B1}$ in Figure \ref{fig:timeout:tx}.

\begin{figure}[hbt!]
	
	\centering
	
	\begin{lstlisting}[
	basicstyle=\small, %or \small or \footnotesize etc.
	linewidth=10cm
	]
<@\texttt{\textcolor{black}{OP\_IF}}@>
  <@\texttt{\textcolor{black}{<$PK_{RA}$>}}@>
<@\texttt{\textcolor{black}{OP\_ELSE}}@>
  <@\texttt{\textcolor{black}{<time> OP\_CHECKSEQUENCEVERIFY OP\_DROP }}@>
  <@\texttt{\textcolor{black}{<$PK_{B}$>}}@>
<@\texttt{\textcolor{black}{OP\_ENDIF}}@>
<@\texttt{\textcolor{black}{OP\_CHECKSIG}}@>
	\end{lstlisting}
	
	\caption{\textit{ Showing the transaction $ST_{B}$ spends to, the keys are reversed for $ST_{A}$.
	}}
	\label{fig:timeout:tx}
\end{figure}

A $HTLC$ is terminated off-chain by creating a new commitment transaction, without $HTLC$ outputs, reflecting the new balance state as both parties knows what would transpire if the transaction was broadcast. In the creation of the new $CT$ both parties disclose both private keys corresponding to the two revocation outputs as a penalty if an old $CT$ was broadcast. Hence most transactions will be kept off chain and only held in case of an uncooperative party.

In this manner Hashed Timelock Contracts may be used without the need for either of the channel parties to trust each other.

Note that key generation and disclosure may be used for $\gls{HTLC}$ in the same manner previously discussed in section \ref{sec:hdw}.

\newpage
\onecolumn

\begin{figure}[!htb]

	\centering
	\includegraphics[width=15cm]{images/flow_htlc.png}

	\caption{\textit{
			Shows the state of transactions after Alice and Bob have agreed to a new \gls{commitment transaction} $CT_{A3}$, $CT_{B3}$. 
			The transactions to the left side can be broadcast by Alice and the ones to the right by Bob. With the signing of the new $CT$s
			Bob and Alice disclose keys for $BRT_{AH1}$, $BRT_{AH2}$, $BRT_{BH1}$, $BRT_{BH2}$ as penalty if an old commitment transaction is broadcast.
		}}
\end{figure}
\newpage
\twocolumn

\section{Multi-hop payments} 

With properly funded channels and Hashed Timelock Contracts in place network wide payments utilizing multiple channels are possible.
The network uses a Gossip protocol\footnote{as per BOLT\#7 routing-gossip in Lightning RFC\cite{repository:lightning:rfc}} where
public channels are broadcast. Each \gls{node} may then find a path through multiple channels and construct a chain of $HTLC$s with decreasing time locks.

Alice wants to pay David but have no direct path to David. She finds a route via Bob and Carol to David as seen in Figure \ref{fig:multihop}.

\begin{figure}[!htb]

	\centering
	\includegraphics[width=7.5cm]{images/multihop.png}

	\caption{\textit{
			Shows a multihop payment from Alice to Dave across three channels $C_{AB}$, $C_{BC}$, $C_{CD}$. It's critical that the $HTLC$s is constructed in the correct order and with decreasing timesteps $TL_{1}$ > $TL_{2}$ > $TL_{3}$.
		}}
	\label{fig:multihop}

\end{figure}

Alice and David agree on size of the payment and David constructs R and shares $H_{R}$ with Alice. Alice then uses $H_{R}$ to construct a $HTLC_{1}$.
After Alice has created $HTLC_{1}$ with $TL_{1}$, Bob is comfortable creating $HTLC_{2}$ with a $TL_{2}$ < $TL_{1}$ as the only way Carol could redeem $HTLC_{2}$ is by disclosing R to Bob. Bob in turn may then disclose R to Alice and redeem $HTLC_{1}$. If the timelocks were improperly set as $TL_{2}$ > $TL_{1}$ the $HTLC_{1}$ may be invalid by the time Carol redeem $HTLC_{2}$ leaving Bob paying Carol but without getting payed by Alice.

\section{Routing fee}

The routing \gls{node}s receive a fee as compensation for displaced channel liquidity, the time value of money and operational cost of bandwidth, computation and electricity. The fee is structured with a base fee $F_{B}$ and a fee $F_{R}$ proportional to satoshis forwarded $AF$. The base fee is always paid independent of payment size while the proportional part is the product of size and fee. The base fee is denominated in microsatoshi (\textbf{μS}) and the proportional in microsatoshi per satoshi forwarded (\textbf{μS / S}). The fee for a payment $P$ is thus:

\[ F_{P} = F_{B} + \Big(AF \dfrac{F_{R}}{1000000}\Big) \]

This fee structure does not discriminate for channel balance and it is difficult to 'discount' transactions that 'balance' the network. Reasons for this is given in Chapter 8.  
