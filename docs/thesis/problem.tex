\subsection{Background}
    \label{sec:background}


\subsection{Aim}
    \label{sec:aim}

This thesis aim to address following three questions:

    \begin{enumerate}
	\item What is the fair discount for a transaction that balances the channel, as
		opposed to an unbalancing one?
	\item When should a new channel be opened, and what node should the new
		channel connect to in order to incentivize routing through the node?
	\item When should a manual rebalancing of the channels occur, as opposed to
		incentivized rebalancing using discounts?
    \end{enumerate}

\subsection{Related work}
    \label{sec:related_work}

	Nothing in the literature that addresses the estimation of the Lightning
	Network Routing Fee has been found. 
	
	A similar problem at first glance, the estimation of the on-chain
	Bitcoin Network(BN) Fee has been covered extensively. However it's not applicable, nor
	is it similar other than domain. The BN Fee problem is more direct since the competing transaction 
	fees(mem-pool) are known at all times and estimating a fee too low can be corrected by \textit{fee bumping}\footnote{Broadcasting a new transaction spending the same UTXO with a higher fee.} or 
	\textit{child-pay-for-parent}\footnote{Using the unconfirmed parent UTXO in a new transaction, a miner must include the parent tx to be able to mine the child tx  and receive the child tx fee.} implementations. Most wallets have a decent fee estimator implemented.
	
	Problems dealing with competing interests are many in the bitcoin domain. Game Theory has mostly been 
	applied to \textit{mining}. E.g. How electricity prices affect mining\cite{singh:dwivedi:strivastava:energy:consumption:mining} and How the introduction of uncle and 
	nephew rewards would affect mining strategies\cite{niu:feng:selfish:mining:ethereum}.
	
	
