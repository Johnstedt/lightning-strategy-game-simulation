\chapter{Problem Description}

\section{Background}
    \label{sec:background}

Cinnober is a provider of IT solutions to the infrastructure providers of the global financial industry, including exchanges and clearing houses. Cinnober has solutions from price discovery, trading, clearing to settlement of financial securities.

\gls{bitcoin} is a peer-to-peer electronic cash system~\cite{nakamoto:bitcoin}. The decentralized nature of the system limits the verification to the weakest node - consequently limiting the transaction capacity. 

Multiple solutions to reduce the burden on the Bitcoin nodes have been suggested~\cite{decker:wattenhofer:duplex, decker:russell:Osuntokun:eltoo, poon:dryja:lightning:network, blockstream:sidechain}. Where the solution receiving most attention is the \gls{Lightning Network} relying on payment channels.

In a payment channel, bitcoin is committed by a Bitcoin transaction and locked up between two parties. Once the transaction is registered, parties in the channel with positive balance may send it to the other party without the need to create a new transaction on the Bitcoin layer. The channel is strictly bidirectional and have a fixed capacity. Each party may settle the channel balance at any point by creating a closing Bitcoin transaction, settling the balance. Usage of payment channels may be very convenient with parties that transact often. 

Many payment channels can be aggregated into a network. There is an ongoing effort to build such a network named The Lightning Network. Where a party without a direct channel to another party could route through multiple channels if enough liquidity exists between them. If such a scheme becomes viable in practice;  only a small fraction of all transactions would need to be settled on the Bitcoin blockchain. Thus increasing the capacity significantly.

How the routing nodes should operate is still vastly unexplored. The node provides liquidity to payment and procures fees whenever his channels are used in a payment route. It is likely that a market for liquid channels will emerge, however strategically well allocated liquidity may earn much more than poorly allocated liquidity. The nodes may chose who to open channels with, what fee to set, how much liquidity should be locked up on each channel and when to re-balance channels.

A simulation environment was implemented, along with operational strategies. Implemented strategies was then be simulated to see how well they perform against each other.

\section{Aim}
    \label{sec:aim}

This thesis aims to address following two questions:

	\begin{enumerate}
		\item How should a \gls{Lightning Network} routing node behave to maximize earnings?
	
		\item What network topology will the \gls{Lightning Network} converge towards as routing nodes becomes increasingly efficient?
		
		\item Is routing a profitable venture? ..
		%\item What network topology will the \gls{Lightning Network} converge towards as routing nodes becomes increasingly efficient in terms of degree distribution, robustness and diameter?
		
	\end{enumerate}
	
\section{Related work}
    \label{sec:related_work}

	Little to nothing in the literature that addresses the Lightning Network Routing Strategies has been found. René Pickardt has made some early attempts to find suitable channel parties with traditional graph heuristics but is yet incomplete and unpublished~\cite{repository:rene:pickard}. Bitmex made an empirical study where different fees were set on Mainnet channels and corresponding returns where retrieved~\cite{bitmex:fee}\footnote{Note that this study was published 27th of Mars, 2019, mid-thesis.} . Further conversation with people in the community confirms that this problem set is known yet largely unexplored.
	
	The actual routing in LN has been researched~\cite{distasi:avallone:cononico:routing} and implementations are currently in use~\cite{repository:clightning, repository:lnd, repository:eclair, repository:lit}. Further efficient sharing of routing tables have been proposed~\cite{gunspan:marco:ant}.
	
	A similar problem at first glance, the estimation of the on-chain Bitcoin Network Fee has been covered extensively~\cite{mosterland:transaction:fee, houy:transaction:fee}. However it's neither applicable, nor is it even similar as a problem set. The Bitcoin Fee problem is far simpler as the competing transaction fees(mem-pool) are known at all times and estimating a fee too low can be corrected by \textit{fee bumping}\footnote{Broadcasting a new transaction spending the same UTXO with a higher fee.} or \textit{child-pay-for-parent}\footnote{Using the unconfirmed parent UTXO in a new transaction, a miner must include the parent tx to be able to mine the child tx and receive the child tx fee.} implementations.
	
	As the Lightning Network is indeed a network and many problems are yet another incarnation of already solved graph problems. A nodes position in a network has been studied and specifically the measurement of Betweenness Centrality, as introduced by Freeman in 1977~\cite{brandes:betweenness:centrality:algorithm}, lays a sound basis for fee estimation. Calculating paths in graphs are as old as the computing field itself with Floyd-Warshall, Johnson~\cite{johnson:shortest:path:sparse:network} and Dijkstras algorithms all being appropriate here. The attributes of the network as strategies emerge have consequences on throughput and robustness have also been researched in the field of topology. Especially that of scale free networks by Barabasi and Albert~\cite{barabasi:albert:emergent:scaling} is of concern here.
	
	Emergence of equilibria as the product of strategies of competing actors is neither new or rare in the fields of evolutionary biology nor Economics. Ideas as the Standard Price Theory, Evolutionary Stable Strategies and more widely Game Theory has aided in the formulation of the thesis and construction of the simulations.
	
	Lastly, \gls{bitcoin} and off-chain proposals have opened a plethora of possible research topics in which plentiful are ongoing and active.  
	
%	Problems dealing with competing interests are many in the bitcoin domain. Game Theory has mostly been 
%	applied to \textit{mining}. E.g. How electricity prices affect mining\cite{singh:dwivedi:strivastava:energy:consumption:mining} and How the introduction of uncle and 
%	nephew rewards would affect mining strategies\cite{niu:feng:selfish:mining:ethereum}.
	
\section{Structure}

This thesis is divided into eight chapters, including this one. The first three chapters give a wide overview of money, Bitcoin and the \gls{Lightning Network}. Chapters 6 and 7 aim to answer the thesis question by formulating, evaluating and simulating routing strategies. The last chapter 8 discusses the results, draws conclusions and elaborates on future work.

\begin{enumerate}
	\item \textbf{Introduction} introduces the thesis.
	\item \textbf{Macroeconomics} gives a wide background to currencies and systems built on trust. 
	\item \textbf{The Bitcoin Network} chapter introduces the basic \gls{bitcoin} architecture and answers how trust is solved and describes Bitcoin script as it allows further development no top of the base protocol.
	\item \textbf{Scaling Bitcoin} describes historical attempts to scale Bitcoin and the merit of a wide array of scaling solutions.
	\item \textbf{The \gls{Lightning Network}} chapter introduces the most fundamental parts of the Lightning Network protocol.
	\item \textbf{Evaluation} reduces the Lightning Network to a manageable problem set, formulates strategies and measurements and suggests how these strategies may be evaluated.
	\item \textbf{Results} show the simulated results of the strategies and their effect on network as a whole in terms of throughput, efficiency and robustness.
	\item \textbf{Discussion and Conclusions} processes the results, draws conclusions and ties them into the wider ecosystem. The consequences and viability of the Lightning Network are elaborated and future work is discussed.
\end{enumerate}