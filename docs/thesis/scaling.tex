\chapter{Scaling bitcoin}

There is no secret that the bitcoin blockchain will not be able to scale on its own. When Nakamoto first published the whitepaper on the cryptographic mailing list his first response was in fact that such a solution won't scale very well\cite{donald:scale}. 

Andreas Antonopoulos has drawn similarities between the scaling of bitcoin with the scaling of the internet:

\begin{displayquote}
	"Bitcoin is failing to scale. If we’re really lucky, bitcoin will continue to fail to scale gracefully for 25 years, just like the internet."\cite{antonopoulos:the:internet:of:money}
\end{displayquote}

His quote in many ways are spot on as Usenet used a \textit{Store-and-forward} method for content to reach every server in the network. \textit{Store-and-forward} does not scale very well and is somewhat similar to the base Bitcoin protocol. The internet moved away from \textit{Store-and-forward} to more direct routing and Bitcoin is too, allowing transactions to happen off-chain.

In this chapter the various scaling alternatives will be discussed.

\section{Block Size limit}

In 2010 Nakamoto reduced the block size limit from 32 to 1 Megabyte as part of two commits. The first commit introduces the new \texttt{BLOCK\_SIZE\_LIMIT} constant\cite{nakamoto:commit:1} and the other commit enforcing the blocks to be under the limit\cite{nakamoto:commit:2}. Nakamoto did not give any explanation as to the reason of this limit. As there was only a few transactions per block at this time a limit reduced the worst case scenario for a DoS attacker filling up the blocks with dummy transactions. 

Increasing the amount of transactions would also increase the imposing cost for the ecosystem by forms of bandwidth, processing, storage and the speed by which blocks propagate through the network. By imposing a block limit the cost of operating a node would be kept low allowing more peers to be able to participate in the network. 

% would come at an increased cost for the ecosystem (bandwidth, processing, and storage for relay nodes, as well as an impact on propagation speed of blocks on the network

% TODO: Math example, compare visa 

\section{Increasing block size limit}

There has been many proposals to increase the block size. Including a handful of BIPs:

\begin{itemize}
	
	\item \textbf{BIP 100} allows the miner to vote on the block limit by encoding a proposed value in the coinbase unlocking script. A 75\% supermajority may increase the block size limit every 2016 blocks. Each period change is limited to only increase by 5\% from the previous period.\cite{bip:0100:dynamic:block:size}
	
	\item \textbf{BIP 101} proposes an immediate increase to 8 megabyte blocks and continue to double every second year. In the two year periods the size would increase at a linear pace based on the block timestamp.\cite{bip:0101:increase:block:size}
	
	\item \textbf{BIP 102} is a simple one time increase to 2 megabytes. The BIP would be triggered if 95\% of the latest blocks signaled support for the upgrade.\cite{bip:0102:increase:2mb}
	
	\item \textbf{BIP 103} aims to increase the block size in accordance with the technology. In practice however it increases the block size with 4.4\% every ~97 day, 17.7\% annually. \cite{bip:0103:increase:with:technology}
\end{itemize}

Most of these BIPs are implemented as a hard forks. Older nodes would regard the bigger blocks as being over the limit and thus invalid.
If poorly executed, without a great majority upgrading the software in time, the network would be forked into two networks with diverging transaction histories. 

% need better name for community worker
\subsection{Contentious Hard Forks}

No proposal to increase the block size has yet received enough support to be activated. This has caused some controversy among community members and multiple have attempted to increase the limit by force. 

One of the earliest attempts to increase the block size was when node Bitcoin XT implemented BIP 101 mentioned above. It failed to reach the miner consensus it required to activate. The BIP 101 was then removed and replaced by a 2 megabyte block limit which forked bitcoin into a new network known as bitcoin classic. Mike Hearn, co-creator of Bitcoin XT along with Gavin Andersen, has made his intention clear arguing that the block size must increase for bitcoin to reach any adoption beyond a fringe niche\cite{hearn:classic}. 

\begin{figure}[!htb]
	\hspace*{-0.7cm} 
	\centering
	\includegraphics[width=8cm]{external/Bitcoin_XT_Nodes.png}
	\caption{\textit{Historic Bitcoin XT node count. As of 2019-02-25 only 3 nodes are still active. Data retrieved from coin.dance\cite{coin:dance}.}}
	\label{fig:xt_nodes}
	\hspace*{2mm} 	
\end{figure}

As figure \ref{fig:xt_nodes} attests, bitcoin XT and classic failed to receive support and is now dead.

\subsection{The Bitcoin Cash saga}

The failures of Bitcoin Classic was just the beginning of the scaling debate. The proposal of Segregated Witness\cite{bip:0141:segwit} was met with strong opposition as it was considered an overtly complicated and risky upgrade when the block limit could simply be increased. 

Segregated Witness is in a way an increase in block space as it moves the unlocking script(witness) from the transaction to its own structure. Since each transaction takes up less space, more transactions fit in each block. The upgrade also addresses malleability, which many off-chain solutions require, and aligns economics incentives with resource costs\cite{antonopoulos:segregated:witness:align:economic:incentives}. Although its possible to implement Lightning without a malleability fix, it increases its viability\cite{song:lightning:malleability}.

The malleability fix also addresses covert \texttt{ASICBOOST} or Antbleed. Covert \texttt{ASICBOOST} utilizes that part of \texttt{SHA256} may be pre-calculated if the last 16 bits remain the same. The last 16 bytes in the bitcoin header include the last 4 bytes of the Merkle root. By finding two Merkle Roots with the same last 4 bytes, allowing for less computation in finding hashes. To find a matching pair of roots, many roots must on average be tried. The space of viable roots is reduces if the transaction witness isn't malleable\cite{song:asicboost} as the witness can't be manipulated to change the transaction and thus the Merkle root. Gregory Maxwell alluded that \texttt{ASICBOOST} had in fact been implemented in ASIC chips and proposed a fix\cite{maxwell:asicboost:fix}. Bitmain, the largest chip manufacture, made a statement 2 days after addressing the accusation\cite{bitmain:response}. There is no way of ultimately proving if covert \texttt{ASICBOOST} was ever used\footnote{In further conversation with Bitmain representative Nishant Sharma at their headquarters in Beijing in December 2017 he stated that they would never use \texttt{ASICBOOST} of moral reasons. \texttt{ASICBOOST} was not longer viable on Bitcoin at that time, but was on Bitcoin cash and other currencies. He also stated that bitmain had been in negotiation to buy the patent from Timo Hanke and Sergio Demian Lerner and had thus filed a patent for \texttt{ASICBOOST} with the Chinese government. The deal ultimately didn't fell through, and while this information is hearsay at best, the patent sparked many rumors at the time and drove opposition and is one reason for the eventual split.}. 

Segregated Witness When BIP148\cite{bip:148:uasf:segwit}, a \textit{user-activated-soft-fork} scheduled for 22 August 2017 to activate Segregated Witness, was getting traction among nodes, the opposing party(including aforementioned Bitmain) decided to hard fork bitcoin into two by raising the block size limit. The hard fork took place on the first of August 2017 and created Bitcoin Cash.

Bitcoin Cash received significant economic support initially, although significantly below the original chain.


-- Chart over price here --


Although support for Bitcoin Cash has dwindled over the following years, it set a precedence of forks may be utilized. During the second half of 2017 many more forks happened similarly to bitcoin cash.

\subsection{The SegWit2x Compromise}

Most forks are simple copies of bitcoin and may be of little interest here. However SegWit2x carry some historic weight worth noting. 
SegWit2x was originally a compromise 


\section{Removing decentralization}

\section{Side chain}

\section{Off-chain scaling}

(payment channels)

Payment channel networks may be constructed in many different ways, a few distinct proposals has been suggested. Decker and Wattenhofer suggested a \_\_ in which \_\_ however \_\_ \cite{decker:wattenhofer:duplex}.  

\section{Rational for a small blocks}
