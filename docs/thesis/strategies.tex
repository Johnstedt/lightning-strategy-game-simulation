\chapter{Strategy Guide}
\label{chapter:strategies}



\section{Simulation Environment}


\section{Strategies}


\section*{Fee Strategies}

The Fee Strategy determine which fee($F_B$, $F_P$) to set to each channel.

\subsection*{random\_fee \texttt{interval\_base} \texttt{interval\_proportional}}

Sets the base- and proportional fee randomly given intervals. This is used as a sanity check benchmark for the other strategies.

\subsection*{title}

\section*{Preferential Attachment Strategies}

The Preferential attachment determine the bias a node use in selecting a node to open a channel with.

\subsection*{random\_attachment}

Selects a random node in the network and opens a channel. It must not be confused with the Erdős–Rényi model as it has a innate age bias.

\subsection*{barabasi-albert}

Selects a node in the network with a probability dependent on degree according to the Barabási-Albert model. The probability is given by:

\[ p_i = \dfrac{k_i}{\sum_{j}^{}k_j}  \]

where $p_i$ is the probability to select $i$ and k is the degree.

\section*{Timing Strategies}
The Timing strategies determine when a node should close a channel.

\subsection*{sanity\_check}

Does not close any channel whatsoever. It is used as a sanity check benchmark for the other strategies.

\subsection*{bankruptcy\_avg\_usage}

Does close a channel if the usage, on average, leads to bankruptcy. A channel is closed if and only if:

\[ \frac{recent\_profit}{channel\_liquidity} < \frac{floor\_bankrupcy}{total\_funding} \]
 
\subsection*{bankruptcy\_avg\_usage\_aggressiveness \texttt{scalar}}

Similar to \textbf{bankruptcy\_avg\_usage} but scaled with an aggressiveness scalar:

\[ \frac{recent\_profit}{channel\_liquidity} < \frac{floor\_bankrupcy}{total\_funding} \texttt{scalar} \]

\[ \texttt{scalar} \in \mathbb{R} \text{ and } 0 < \texttt{scalar}  \]

A \texttt{scalar} greater than 1 may be called a \textit{hawkish} strategy and one lesser than 1 may be called \textit{dovish}.

\section*{Funding strategies}

The funding strategy determine how much the node has available to fund channels.

\subsection{funding \texttt{satoshis}}

Funds the node with \texttt{satoshis}. Although relative a node with large quantity may be called a \textit{rich} strategy and
one with small quantity a \textit{poor} strategy. It may seem dissonant to reefer to wealth as a strategy but remember 
that profit is relative to funding and one operator may have many nodes instead of one large if a \textit{poor} strategy
has higher \textit{fitness}. Further beetle size is similarly referred to as a strategy in EGT.


\section*{Re-balance}

\subsection*{sanity\_check}

Does not re-balance channels whatsoever. It is used as a sanity check benchmark for the other strategy