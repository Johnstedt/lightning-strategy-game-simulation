\chapter{Strategy Guide}
\label{chapter:strategies}

This appendix contains a short summary of each implemented strategy and tunable environment variable. For examples of how simulations may be created a set of configured presets, used to generate the results, are available at \texttt{https://github.com/Johnstedt/lightning-strategy-game-simulation/tree/master/src/presets}.

\section{Simulation Environment}

\subsection*{Bankruptcy} 


\section{Strategies}

Following is a list of all implemented strategies and their arguments, 

\section*{Fee Strategies}

The Fee Strategy determine which fee($F_B$, $F_P$) to set to each channel.

\subsection*{random\_fee \texttt{interval\_base} \texttt{interval\_proportional}}

Sets the base- and proportional fee randomly given intervals. This is used as a sanity check benchmark for the other strategies.

\subsection*{probability\_model \texttt{path-to-json}}

Each edge price is set following a json with four attributes. An array of the range of base fees accompanied another array with the probability for each price. Similarly an array of the range of proportional fees and an array with their probability. These models may be generated with \texttt{price\_strategies.py}, with simulates a simulation and selects a central edge, calculates the optimal fee curve from chapter 6, and creates a model with probability correlating with expected relative profit. The price model may be used in a new simulation generating a new model. This process can be repeated many times.  

\section*{Preferential Attachment Strategies}

The Preferential attachment determine the bias a node use in selecting a node to open a channel with.

\subsection*{random\_attachment}

Selects a random node in the network and opens a channel. It must not be confused with the Erdős–Rényi model as it has an innate age bias.

\subsection*{barabasi-albert}

Selects a node in the network with a probability dependent on degree according to the Barabási-Albert model. The probability is given by:

\[ \Pi(i) = \dfrac{k_i}{\sum_{j}^{}k_j}  \]

where $\Pi(i)$ is the probability to select $i$ with degree $k_i$.

\subsection*{hassan\_islam\_haque}

Randomly chooses a node and then selects one of its neighbors randomly according to the Mediation-driven attachment model. The probability is given by:

\[ \Pi(i) = \dfrac{1}{N} \bigg\lbrack \dfrac{1}{k_1} + \dfrac{1}{k_2} + ... + \dfrac{1}{k_{k_1}} \bigg\rbrack = \dfrac{\sum_{j=1}^{k_i}\dfrac{1}{k_j}}{N} \]

where $\Pi(i)$ is the probability to select $i$ with degree $k_i$.

\subsection*{inverse\_barabasi\_albert}

Selects a node in the network with a probability inversely dependent on degree. The probability is given by:

\[ \Pi(i) =  \dfrac{\dfrac{1}{k_i}}{\sum_{j}^{n} \dfrac{1}{k_j}} \]
 
where $\Pi(i)$ is the probability to select $i$ with degree $k_i$.

\section*{Timing Strategies}
The Timing strategies determine when a node should close a channel.

\subsection*{sanity\_check}

Does not close any channel whatsoever. It is used as a sanity check benchmark for the other strategies.

\subsection*{bankruptcy\_avg\_usage}

Does close a channel if the usage, on average, leads to bankruptcy. A channel is closed if and only if:

\[ \frac{recent\_profit}{channel\_liquidity} < \frac{floor\_bankrupcy}{total\_funding} \]
 
\subsection*{bankruptcy\_avg\_usage\_aggressiveness \texttt{scalar}}

Similar to \textbf{bankruptcy\_avg\_usage} but scaled with an aggressiveness scalar:

\[ \frac{recent\_profit}{channel\_liquidity} < \frac{floor\_bankrupcy}{total\_funding} \texttt{scalar} \]

\[ \texttt{scalar} \in \mathbb{R} \text{ and } 0 < \texttt{scalar}  \]

A \texttt{scalar} greater than 1 may be called a \textit{hawkish} strategy and one lesser than 1 may be called \textit{dovish}.

\section*{Funding strategies}

The funding strategy determine how much the node has available to fund channels.

\subsection*{funding \texttt{[satoshis]}}

Funds the node with \texttt{satoshis}. Although relative a node with large quantity may be called a \textit{rich} strategy and
one with small quantity a \textit{poor} strategy. It may seem dissonant to reefer to wealth as a strategy but remember 
that profit is relative to funding and one operator may have many nodes instead of one large if a \textit{poor} strategy
has higher \textit{fitness}. Further beetle size is similarly referred to as a strategy in EGT.


\section*{Re-balance}

The re-balancing strategy determine if it is worth to re-balance a pair of channels by routing a payment to oneself.

\subsection*{sanity\_check}

Does not re-balance channels whatsoever. It is used as a sanity check benchmark for the other strategy.

\subsection*{linear\_displacement \texttt{[depth]} \texttt{[ratio]}}

Does a \textit{breath-first-search} for cycles with \texttt{[depth]}. If the \texttt{[ratio]} given by an Utility(\ref{sec:linear:displacement}) as,

\[ U(c) = \begin{cases} 
\dfrac{2|(B_{1S} - B_{1M})|}{f_c(|B_{1S} - B_{1M}|)}  & if~B_{1S} - B_{1M} < B_{2M} - B_{2S} \\ 
\\
\dfrac{2|(B_{2M} - B_{2S})|}{f_c(|B_{2M} - B_{2S}|)}  & otherwise
\end{cases} \]

where $B_{1S}$, $B_{2S}$ are the states of the out and in channels and $B_{1M}$, $B_{2M}$ their middles and $f$ is the total fee for the cycle $c$. 

\[f_c(L) = \sum_{i=1}^{n} B_i + P_iL \]

with $B_i$ is the base fee, $P_i$ is the proportional fee for node $i$ in path $c$.

\subsection*{edge\_biased\_displacement \texttt{[depth]} \texttt{[ratio]} \texttt{[scalar]}}

Does a \textit{breath-first-search} for cycles with \texttt{[depth]}. If the \texttt{[ratio]} given by a biased to aforementioned Utility function as,

\[ U'(c) = U(c) \bigg(1,5 - \big(1 - \dfrac{B_{1S}}{2B_{1M}}\big) \bigg)^s \bigg(1,5 - \big(\dfrac{B_{2S}}{2B_{2M}}\big)\bigg)^s  \]

where $s$ is a scalar given as argument \texttt{[scalar]}
