\subsection{Background}
    \label{sec:background}


\subsection{Aim}
    \label{sec:aim}

    This report aims to address three questions:

    \begin{enumerate}
        \item What effect does the uneven IP distribution have on
            the pre-calibration?
        \item Does assigning equal weights to equal sensor areas
            during the calibration yield a better calibration?
        \item Do the answers to these questions depend on whether
            the calibration object is planar?
    \end{enumerate}

\subsection{Related work}
    \label{sec:related_work}

    Little in the literature has been found that addresses the issue
    with uneven calibration directly.
    \citet{jung_lee_yoon_inverse_mapping} addresses the problem with
    polynomials being poor extrapolators by making assumptions 
    about the characteristics of the radial distortion function.
    These assumptions improved the calibration of wide angle lenses,
    using a calibration software adapted for general camera lenses.

    When estimating the quality of a calibration,
    several metrics have been proposed.
    \citet{Rabbani2007:Integrated} and 
    \citet{Grussenmeyer2008:Comparison}
    compared 3D reconstruction techniques by fitting a model object,
    e.g., a plane or a cylinder to points estimated by one technique and
    examined the distribution of the orthogonal distances between the object and 
    the points estimated by another technique.
    \citet{Boehler2003:Investigating} compared the absolute distances between 
    the corresponding estimated points for different calibration techniques.

    Some authors estimate whether the resulting 
    calibrations are consistent with each other, 
    \citet{Rabbani2007:Integrated} compared the maximal distance between
    the fitted models for the different techniques.
    \citet{Dickscheid2008:benchmarking} and \citet{Labe2008:Quality}
    examined whether the estimated accuracy for the estimated parameters
    is similar enough.

    \citet{Rabbani2007:Integrated} and \citet{Fraser1995:Multi_sensor_self_calib} 
    compared the estimated measurement uncertainty to the accuracy stated by
    the instrument manufacturer.

    \citet{Fraser1995:Multi_sensor_self_calib} and \citet{Labe2008:Quality}
    estimated the standard deviations of the estimated IO, EO and OP.
    \citet{Fraser1995:Multi_sensor_self_calib} used the 
    \textit{root-mean-square} (RMS) standard deviation of the OP as a 
    measure of quality while \citet{Rabbani2007:Integrated} used the 
    mean absolute standard deviation.
    Furthermore, \citet{Fraser1995:Multi_sensor_self_calib} separated 
    the $XY$ uncertainty from the $Z$ uncertainty since the $Z$ uncertainty
    is usually much larger for weak networks, e.g., in aerial imagery.
    Additionally, \citet{Fraser1995:Multi_sensor_self_calib} 
    estimated and the anticipated uncertainty to detect any 
    systematic error in the calibration.
